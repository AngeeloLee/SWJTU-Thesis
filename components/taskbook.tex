% !TeX root = ../paper.tex

% 设置任务书的一些信息
\swjtu{
    task@start  = {2020年12月11日},
    task@end    = {2021年5月29日},
    task@mean   = {
        LaTeX(LATEX,音译“拉泰赫”)是一种基于ΤΕΧ的排版系统,由美国计算机学家莱斯利·兰伯特(Leslie Lamport)在20世纪80年代初期开发,
        利用这种格式,即使使用者没有排版和程序设计的知识也可以利用TeX所提供的强大功能,生成很多具有书籍质量的印刷品。
        LaTeX对于生成复杂表格和数学公式表现得尤为突出,因此它非常适用于生成高印刷质量的科技和数学类文档。
        这个系统同样适用于生成从简单的信件到完整书籍的所有其他种类的文档。
        LaTeX使用TeX作为它的格式化引擎,当前的版本是LaTeX2ε。
        目前我校本科生的毕业论文模板还未有LaTeX的版本,此模板为自制的模板,仍有许多疏漏,但目的在于学习和使用LaTex。
    },
    task@tasks  = {
        了解Latex及其相关知识,
        了解常见的Latex文档类,
        熟悉其他学校的latex模板,
        参照word模板设计自己的模板模块,
        编码实现模板并进行功能测试,
        撰写毕业设计论文
    },
    task@target = {
        本论文支撑本专业以下毕业要求的达成:
        (1)能够通过查阅和分析文献,为计算机系统及工程的问题求解寻找方案,并认识到所求解的问题具有多种可能的解决途径(指标点2.3);
        (2)能够针对特定需求确定目标,设计计算机系统框架、组成模块,合理组织/存储数据,基于适当的模型进行系统设计与实现,并体现一定的创新意识(指标点3.3);
        (3)能够在解决方案中从技术、非技术(如经济、社会、健康、安全、法律、文化以及环境等)角度,对设计方案的可行性进行评价和分析(指标点3.4);
        (4)能够采用科学方法对计算机系统及工程问题进行研究,通过实验对比、文献综合、归纳整理得到合理有效结论,并对其进行规范表述(指标点4.3);
        (5)能够利用开发环境和工具,对计算机系统及工程问题进行模拟仿真和数据分析(指标点5.3);
        (6)能识别、分析、评价特定需求的计算机系统在设计和实现中对社会、健康、安全、法律以及文化的影响,并明确自己应承担的责任(指标点6.2);
        (7)能够评价计算机系统设计、开发、运行和维护对环境保护和社会持续发展的影响(指标点7.2);
        (8)能够通过口头、文稿、图表等方式、陈述和表达自己的观点,能够就计算机系统及工程问题与同行和相关人员进行交流(指标点10.1);
        (9)能够根据对工作内容和过程的记录与整理,撰写技术报告和设计文稿、陈述发言或回应质询(指标点10.2);
        (10)了解计算机系统工程管理原理与经济决策方法,理解计算机系统项目的组织模式和实施过程,掌握项目管理原理和内容(指标点11.1);
        (11)正确认识自主学习的必要性和重要性,认识到本专业是一个发展迅速的学科,具有自主学习和终身学习的意识(指标点12.1);
        (12)具备自主学习新技术和新方法的能力,能够通过学习不断提高、适应信息技术和职业的发展(指标点12.2)。
    },
    task@steps  = {
        了解Latex及其相关知识,
        了解常见的Latex文档类,
        设计模板的各个模块,
        编码实现模板并进行功能测试,
        撰写毕业设计论文
    }
}

\maketaskbook